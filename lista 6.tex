\documentclass[11pt,a4paper]{article}

\usepackage{epsfig}
\usepackage{multicol}

\usepackage[utf8]{inputenc}
\usepackage[brazil]{babel}
\usepackage{fancyheadings}
\usepackage{amsmath}
\usepackage{calrsfs}
\usepackage{enumerate}
\usepackage{enumitem}   
\DeclareGraphicsExtensions{.png,.pdf}
\usepackage{amsmath, amsfonts, amssymb}
\usepackage{esint}
\usepackage{graphicx}
\usepackage{multicol}
\usepackage{tasks}
\usepackage[utf8]{inputenc}
\usepackage{mathrsfs} % Transformada de Laplace
\usepackage{indentfirst}

% As margens
\setlength{\textheight}{24.0cm}
\setlength{\textwidth}{17.5cm}
\setlength{\oddsidemargin}{2.0cm} % Margens reais desejadas
\setlength{\evensidemargin}{2.0cm} % 2+17.5+1.5=21cm (largura A4)
\setlength{\topmargin}{1.5cm} % 1.5+1.6+1.0+24.0+1.6=29.7cm
\setlength{\headheight}{1.6cm} % (altura A4)
\setlength{\headsep}{1.0cm}
\setlength{\columnsep}{1.5cm} % Coluna = 8cm ((17.5-1.5)/2)
\addtolength{\oddsidemargin}{-1in}
\addtolength{\evensidemargin}{-1in}
\addtolength{\topmargin}{-1in}
\setlength{\footskip}{0.0cm}


% Novos comandos
\newcommand{\limite}{\displaystyle\lim}
\newcommand{\integral}{\displaystyle\int}
\newcommand{\somatorio}{\displaystyle\sum}
\newcommand{\mat}[1]{\mbox{\boldmath{$#1$}}} 

\pagestyle{fancy}


\usepackage{lipsum}

\lhead{
\includegraphics[width=1cm]{brasao.png}
}

\rhead{ 
\sc\textbf{U}niversidade \textbf{F}ederal do \textbf{C}eará\\
Campus Quixadá\\ Monitoria de Cálculo II, III e EDO}

\cfoot{}

\begin{document}

	\begin{center}
		\Large Lista 6 - Campos vetoriais não conservativos e Teorema de Green
	\end{center}
	

	\begin{enumerate}
	
	\item Uma partícula é transportada do ponto inicial $A(0 \textrm{,}\ 0 \textrm{,}\ 0)$ até o ponto final $B(1 \textrm{,}\ 1 \textrm{,}\ 1)$ por uma força $\vec{F}$ dada por
  $$\vec{F} = y\vec{i} - x\vec{j} + 3z\vec{k}$$
  
  Determine o trabalho realizado pela força $\vec{F}$ para levar a partícula de $A$ a $B$ ao longo das curvas abaixo. Todas as unidades são do SI.
  
      \begin{enumerate}[label=(\roman*)]
       \item Reta dada por $\vec{r} = t\vec{i} + t\vec{j} + t\vec{k}$.
       \item Curva dada por $y = x^2$, $z = x^3$, de $x = 0$ a $x = 1$.
      \end{enumerate}
 
	\item Uma partícula é transportada do ponto inicial $A(0 \textrm{,}\ 0 \textrm{,}\ 0)$ até o ponto final $B(1 \textrm{,}\ 1 \textrm{,}\ 1)$ por uma força $\vec{F}$ dada por
  $$\vec{F} = 2x\vec{i} - 4x^2\vec{j} + 3yz\vec{k}$$
  
  \begin{enumerate}[label=(\roman*)]
       \item Reta dada por $\vec{r} = t\vec{i} + t\vec{j} + t\vec{k}$.
       \item Curva dada por $y = x^2$, $z = x^3$, de $x = 0$ a $x = 1$.
       \item Segmentos de reta de $A(0 \textrm{,}\ 0 \textrm{,}\ 0)$ a $A(1 \textrm{,}\ 0 \textrm{,}\ 0)$, depois a $A(1 \textrm{,}\ 1 \textrm{,}\ 0)$ até chegar em $A(1 \textrm{,}\ 1 \textrm{,}\ 1)$.
      \end{enumerate}
      
	\item Uma força dada por 
	$$\vec{F} = -2y \vec{i} + 2x\vec{j} + z\vec{k}$$ 
	
	é aplicada sobre uma partícula de massa $m$ constante. Determine o trabalho realizado por essa força quando a partícula executa uma volta completa numa elipse descrita por 
	$$\vec{r}(t) = a\cos t\vec{i} + b\sin t\vec{j}$$ 
	onde $a$ e $b$ são constantes positivas e $t \in [0 \textrm{,}\ 2\pi]$
	
	\item Um campo elétrico dado por
	$$\vec{E} = xy \vec{i} - 3xz\vec{j} + 4y^3z^2\vec{k}$$ 
	
	existe numa certa região, situada no vácuo. Todas as unidades são do SI. Responda ao ao seguinte:
	
	\begin{enumerate}[label=(\roman*)]
       \item Determine a densidade volumétrica de carga na região.
       \item Verifique se o campo elétrico é não conservativo.
       \item Calcule a fem produzida num circuito fechado formado pelos segmentos de reta definidos entre os pontos $A(4 \textrm{,}\ 0 \textrm{,}\ 0)$, $B(0 \textrm{,}\ 0 \textrm{,}\ 2)$, $C(0 \textrm{,}\ 3 \textrm{,}\ 0)$.
      \end{enumerate}
      
     \item Um campo elétrico numa dada região cilíndrica é dado, em coordenadas cilíndricas, por
     $$\vec{E} = \displaystyle\dfrac{3}{\rho^2}\hat{\theta}$$
     
     onde $\rho \leq 5$ e $-5 \leq z \le1 5$. Existe na região um circuito circular de raio $R = 2$ m situado no plano $xy$ e centrado na origem. Esse circuito é formado por fios condutores cuja resistência elétrica total vale $R = 5 \Omega$. Determine a potência elétrica dissipada pelo circuito quando o campo elétrico é ligado. 
     
     \item O campo magnético numa dada região é dado por (em unidades do SI)
     $$\vec{B} = 2y \vec{i} - 5x\vec{j}$$
     
     Determine a circuitação magnética ao longo de um circuito quadrado de vértices $A(0 \textrm{,}\ 0 \textrm{,}\ 0)$, $B(0 \textrm{,}\ 2 \textrm{,}\ 0)$, $C(2 \textrm{,}\ 2 \textrm{,}\ 0)$ e $D(2 \textrm{,}\ 0 \textrm{,}\ 0)$, no sentido $ABCDA$. 
     
     \item Determine a integral 
     
     $$\displaystyle\oint_C (2x^3 - y^3) \ dx + (x^3 + y^3) \ dy$$
     
     onde $C$ é dado pela circunferência descrita por $x^2 + y^2 = 1$. Resolva a integral por dois métodos diferentes.
     
     \item Mostre que a área delimitada por uma curva $C$ plana pode ser obtida por meio de
     $$A = \displaystyle\dfrac{1}{2}\oint_C (x \ dy - y \ dx)$$ 
     
     \item Calcule $\displaystyle\int_C x^4 \ dx + xy \ dy$, onde $C$ é a curva triangular constituída pelos segmentos de reta de $(0 \textrm{,}\ 0)$ a $(1 \textrm{,}\ 0)$, de $(1 \textrm{,}\ 0)$ a $(0 \textrm{,}\ 1)$, e de $(0 \textrm{,}\ 1)$ a $(0 \textrm{,}\ 0)$.
     
     \item Calcule $\displaystyle\oint_C(3y - e^{\sin x}) \ dx + (7x + \sqrt{y^4 + 1}) \ dy$, onde $C$ é o círculo $x^2 + y^2 = 9$.
     
\item Calcule $\displaystyle\ointctrclockwise_C y^2 \ dx + 3xy \ dy$, onde $C$ é o limite da região semianular $D$ contida no semiplano superior entre os círculos $x^2 + y^2 = 1$ e $x^2 + y^2 = 4$.

\item Calcule a integral de linha por dois métodos: 

\begin{enumerate}
\item diretamente e 
\item utilizando o Teorema de Green.
\end{enumerate}


\begin{enumerate}[label=(\roman*)]
       \item $\displaystyle\ointctrclockwise_C (x - y) \ dx + (x + y) \ dy$, $C$ é o círculo com centro na origem e raio 2.
       \item $\displaystyle\ointctrclockwise_C xy \ dx + x^2 \ dy$, $C$ é o retângulo com vértices $(0 \textrm{,}\ 0)$, $(3 \textrm{,}\ 0)$, $(3 \textrm{,}\ 1)$ e $(0 \textrm{,}\ 1)$
       \item $\displaystyle\ointctrclockwise_C x^2y^2 \ dx + xy \ dy$, $C$ consiste no arco da parábola $y = x^2$ de $(0 \textrm{,}\ 0)$ a $(1 \textrm{,}\ 1)$ e os segmentos de reta de $(1 \textrm{,}\ 1)$ a $(0 \textrm{,}\ 1)$ e de $(0 \textrm{,}\ 1)$ a $(0 \textrm{,}\ 0)$. 
\end{enumerate}
      
\item Uma partícula inicialmente no ponto $(-2 \textrm{,}\ 0)$ se move ao longo do eixo $x$ para $(2 \textrm{,}\ 0)$, e então ao longo da semicircunferência $\sqrt{4 - x^2}$ até o ponto inicial. Utilize o Teorema de Green para determinar o trabalho realizado nessa partícula pelo campo de força $$\vec{B} = x\vec{i} + x^3\vec{j} + 3xy^2\vec{k}$$. 

\item Seja $D$ a região limitada por um caminho fechado simples $C$ no plano xy. Utilize o Teorema de Green para demonstrar que as coordenadas do centroide $(\bar{x} \textrm{,}\ \bar{y})$ de $D$ são

$$\bar{x} = \displaystyle\dfrac{1}{2A}\ointctrclockwise_C x^2 \ dy \quad \quad \quad \bar{y} = - \displaystyle\dfrac{1}{2A}\ointctrclockwise_C y^2 \ dx$$ onde $A$ é a área de $D$.

\item Uma lâmina plana com densidade constante $\rho(x,y) = \rho$ ocupa uma região do plano xy limitada por um caminho fechado simples $C$. Mostre que seus momentos de inércia em relação aos eixos são

$$I_x = - \displaystyle\dfrac{\rho}{3}\ointctrclockwise_C y^3 \ dx \quad \quad \quad I_y = \displaystyle\dfrac{\rho}{3}\ointctrclockwise_C x^3 \ dy$$ 
	
\end{enumerate}		
	
\end{document}