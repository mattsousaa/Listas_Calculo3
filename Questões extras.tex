\documentclass[11pt,a4paper]{article}

\usepackage{epsfig}
\usepackage{multicol}

\usepackage[utf8]{inputenc}
\usepackage[brazil]{babel}
\usepackage{fancyheadings}
\usepackage{amsmath}
\usepackage{calrsfs}
\usepackage{enumerate}
\DeclareGraphicsExtensions{.png,.pdf}
\usepackage{amsmath, amsfonts, amssymb}
\usepackage{esint}
\usepackage{graphicx}
\usepackage{multicol}
\usepackage{tasks}
\usepackage[utf8]{inputenc}
\usepackage{mathrsfs} % Transformada de Laplace
\usepackage{indentfirst}

% As margens
\setlength{\textheight}{24.0cm}
\setlength{\textwidth}{17.5cm}
\setlength{\oddsidemargin}{2.0cm} % Margens reais desejadas
\setlength{\evensidemargin}{2.0cm} % 2+17.5+1.5=21cm (largura A4)
\setlength{\topmargin}{1.5cm} % 1.5+1.6+1.0+24.0+1.6=29.7cm
\setlength{\headheight}{1.6cm} % (altura A4)
\setlength{\headsep}{1.0cm}
\setlength{\columnsep}{1.5cm} % Coluna = 8cm ((17.5-1.5)/2)
\addtolength{\oddsidemargin}{-1in}
\addtolength{\evensidemargin}{-1in}
\addtolength{\topmargin}{-1in}
\setlength{\footskip}{0.0cm}


% Novos comandos
\newcommand{\limite}{\displaystyle\lim}
\newcommand{\integral}{\displaystyle\int}
\newcommand{\somatorio}{\displaystyle\sum}
\newcommand{\mat}[1]{\mbox{\boldmath{$#1$}}} 

\pagestyle{fancy}


\usepackage{lipsum}

\lhead{
\includegraphics[width=1cm]{brasao.png}
}

\rhead{ 
\sc\textbf{U}niversidade \textbf{F}ederal do \textbf{C}eará\\
Campus Quixadá\\ Monitoria de Cálculo II e III}

\cfoot{}

\begin{document}

	\begin{center}
		\Large Questões extras - Cálculo III
	\end{center}
	

	\begin{enumerate}
	
\item Os geólogos estudam como as cadeias de montanhas foram formadas e estimam o trabalho
necessário para elevá-las em relação ao nível do mar. Na questão abaixo é solicitado que você use a integral tripla para calcular o trabalho realizado na formação do Monte Fuji, no Japão.

Quando estudam a formação de cordilheiras, os geólogos estimam a quantidade de trabalho necessária para erguer uma montanha a partir do nível do mar. Considere uma montanha que tenha essencialmente o formato de um cone circular reto. Suponha que a densidade do material na vizinhança de um ponto $P$ seja $g(P)$ e a altura seja $h(P)$.

\begin{enumerate}
\item Determine a integral definida que representa o trabalho total exercido para formar a montanha.

\item Assuma que o monte Fuji no Japão tenha o formato de um cone circular reto com raio de $19000$ $m$, altura de $3800$ $m$ e densidade constante de $3200$ $kg/m^3$. Quanto trabalho foi feito para formar o monte Fuji se a terra estivesse inicialmente ao nível do mar? 
\end{enumerate}

\item Na física o estudo de oscilações é muito importante e está muito presente dentro do nosso dia a dia. Um dos problemas mais clássicos que temos é a descrição vibratória de fenômenos oscilatórios, como a vibração de uma das cordas de um violão, por exemplo. Imagine uma corda presa em suas extremidades. Essa corda quando está relaxada possui comprimento L. Alguém a vibra de forma que seu movimento é de forma vibratória e estacionária. Mostre que com as informações abaixo você pode demonstrar a natureza matemática de uma corda em vibração. No final de tudo, esclareça suas conclusões e demonstre o significado das contas. Cálculo com número não é significado. Interprete. 
	
	Uma corda vibra de acordo com uma função $u = u(x,t)$, onde $t$ é o tempo e $x$ é a posição que a onda se encontra do eixo x. Imagine uma corda vibrando ao longo de um plano cartesiano $xy$, onde o eixo y é equivalente a $u$ e o eixo x é equivalente ao tempo. Temos condições de fronteira que modelam o nosso problema, e podem ser definidas como:
	$$u(x,y) = 
		\begin{cases}
			u_{tt} = c^2u_{xx} \\
			u(0,t) = 0 \\
			u(L,t) = 0 \\
			u(x,0) = f(x) \\
			u_t(x,0) = g(x)
		\end{cases}
	$$
	
	$u_{tt} = c^2u_{xx}$ é a equação que descreve movimentos oscilatórios em geral. O método de resolução é por separação de variáveis. Imagine que $u(x,t)$ pode ser decomposta em outras duas funções:
	$$u(x,t) = \phi (t) \psi (t) $$
	A ideia é fazer por partes. Na equação acima você pode usar:
	$$u_{tt} = \phi (x) \psi ''(t) $$
	$$u_{xx} = \psi (t) \phi ''(x) $$
	
	Utilize cada condição de fronteira nas 2 equações acima. Mostre que através disso, podemos descrever $u(x,t)$ como:
	
	$$u(x,t) = \displaystyle\frac{8a}{\pi ^2} \somatorio_{n=1}^{\infty} \displaystyle\frac{(-1)^n}{(2n + 1)^2} \cos \displaystyle\left(\frac{(2n + 1)\pi t}{2}\right) \sin \displaystyle\left(\frac{(2n + 1)\pi x}{2}\right)$$

\end{enumerate}
		
	
\end{document}